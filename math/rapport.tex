\documentclass{article}

\usepackage{amsmath, amssymb, cancel}
\usepackage{verbatim}
\usepackage[utf8]{inputenc}

\title{Dice trading}
\author{François Medina}

\newcommand{\dice}{\{1,\dots,6\}}
\renewcommand{\P}{\mathbb P}
\newcommand{\getoutput}[1]{\verbatiminput{data/output-#1.txt}}

\begin{document}
    \maketitle
    \section*{Question 1}
    \paragraph{(a)}
    This dynamical system is defined by the following elements :
    \begin{itemize}
        \item For each time step $t$,
            a dice count $d_t$ and accumulated score $s_t$
        \item At each step, the possibility (boolean $b_t$)
            to trade off five points for a die
        \item The system follows the following dynamic :
            \begin{align}
                s_0 &= 0,\quad d_0 = 1\\
                b_t &\leq s_t - 5\\
                d_{t+1} &= d_t + b_t\\
                s_{t+1} &= s_t + \max_{1\leq i\leq d_t}X_i - 5b_t
            \end{align}
            Where $X_i$ are iid of uniform law on $\dice$.
    \end{itemize}
    As defined in \texttt{model.py} we hold game rules
    (price of a new dice, number of turns, and upcoming extensions)
    in the structure \texttt{Game}, and the game state in \texttt{State}.
    This allows us to encode each rule set in a simple \texttt{yaml} file.

    \paragraph{(b)}
    In order to easily choose which strategies to use on a given run,
    we use a registration system through the \texttt{Strategy} class.
    We can define a strategy as a standalone function taking the game,
    step and state to return a boolean decision, with a \texttt{@buy}
    (or later \texttt{@sell}) decorator for registration.
    We can also directly use a class inheriting from \texttt{Strategy},
    which will hold game rules as an instance attribute and provide a
    \texttt{buy} method taking just the step and state as arguments.

    In practice the standalone function will be turned into a
    \texttt{Strategy} subclass, and these will be instantiated with
    game rules by the simulator.

    The default behavior for the player is to not do anything.
    We call this \texttt{passive} for future reference.
    A simple heuristic would be to value a new dice by
    the difference in maximum roll expectancy multiplied by the remaining
    playing time, to which we subtract its price.
    We call this the \texttt{basic} strategy.

    Any decision that does not respect game rules will be ignored by the simulator anyways.


    \paragraph{(c)}
    A \texttt{Simulator} instance takes the game rules as its sole
    attribute, and provides a \texttt{run} method that takes a
    sample size and \texttt{Strategy} instance.
    It then plays the game the required number of times with the given strategy.
    This is sped up by multiprocessing, through the constant \texttt{CORES},
    to be adjusted to the hardware the script runs on.
    Finally, the method returns a 95\% confidence interval for the average
    score obtained with the strategy.

    We get the following results with the aforementioned strategies :
    \getoutput{1c}


    \paragraph{(d)}
    Let $X_1\dots X_N$ with $N\in\mathbb N$ be iid of uniform law on $\dice$.\\
    Let $M_N=\max_{i=1}^NX_i$ , we have :
    \begin{equation}
        \forall k\in\dice,\quad
        \mathbb{P}(M_N\leq k)=\P\left(\cup_{i=1}^NX_i\leq k\right)
        =\left(\frac{k}{6}\right)^N
    \end{equation}
    Hence the following distribution for $M_N$ :
    \begin{equation}
        \forall k\in\dice,\quad
        \mathbb{P}(M_N=k)=p^N_k:=\frac{1}{6^N}\left(k^N-(k-1)^N\right)
    \end{equation}
    We can therefore simulate $\max_{i=1}^NX_i$ with :
    \begin{equation}
        M_N'=\min\left\{
            k\in\dice,\quad6^M\sum_{i=1}^kp^N_i>\xi
        \right\}
    \end{equation}
    Where we now have exogenous noise $\xi$ following a uniform distribution on
    $\{1,\dots,6^M\}$ with $M$ greater than any possible $N$
    ($M=10\times6/5=12$ would work here).

    Indeed we have :
    \begin{align}
        \mathbb P(M_N'=k)&=
        \mathbb P\left(6^M\sum_{i=1}^{k-1}p^N_i\leq\xi<6^M\sum_{i=1}^kp^N_i\right)\\
                         &=\frac{\cancel{6^M}p^N_k}{\cancel{6^M}}\\
                         &=\mathbb P(M_N=k)
    \end{align}

    \paragraph{(e-f)}
    With dynamic programming we find the optimal value $V_0=37.62$.
    The associated strategy is : "buy a second dice within the first 5 turns".
    We get the following results from the simulator, with 95\% confidence intervals
    (where \texttt{dynamic} is the solver's solution while \texttt{optimal\_1e}
    is the interpretation described above):
    \getoutput{1e}

    \paragraph{(h)}
    With no dice limit we get the following results :
    \getoutput{1h}
    This means the limit of three dice is ineffective with this set of rules.

    \section*{Question 2}
    \paragraph{(a)}
    The player now has the ability to sell a dice to double the score from a roll,
    if he has at least two dice. This decision now also depends on the dice roll.
    It is implemented like the buying decision with this extra argument,
    through the \texttt{sell} decorator or a \texttt{Strategy} method of the same name.\\
    We consider the following algorithm, defining for each step $t$, score $s$, dice count $d$
    and dice roll $r\in\dice$ :
    \begin{itemize}
        \item $V_{t,s,d}$ and $V'_{t,s,d,r}$ the expected score before and after the roll
        \item $B_{t,s,d}$ and $S_{t,s,d,r}$ the decisions whether to buy and sell a dice
    \end{itemize}
    The algorithm follows the following dynamic :
    \begin{align}
        V_{T,s,\cdot} &= s\\
        V'_{t,s,d,r} &= \max
        \begin{cases}
            V_{t+1,s+r,d}\\
            V_{t+1,s+2r,d-1}\quad\text{if }d\geq2
        \end{cases}\\
        V_{t,s,d} &= \max
        \begin{cases}
            \sum_{r=1}^6p^d_rV'_{t,s,d,r}\\
            \sum_{r=1}^6p^d_rV'_{t,s-5,d+1,r}\quad\text{if }d<5,s > 6
        \end{cases}
    \end{align}
    With $B$ and $S$ being the argmax for each decision.\\
    This gives us the following results :
    \getoutput{2a}

    \paragraph{(b)}
    With no limit we have :
    \getoutput{2b}

    \paragraph{(c)}
    With dice liquidation at the end we get :
    \getoutput{2c}
\end{document}
